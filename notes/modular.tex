
\section{Modular invariance}
Up until now we considered only CFTs on the infinite plane/Riemann sphere and at the exact fixed point. In that case the Hilbert space is a tensor product of a holomorphic and (completely decoupled) anti-holomorphic part. These parts don't interact and can this in principle describe completely different theories. In this section we will consider CFTs on the torus, still at the fixed point, in which case the two parts will need to be coupled in such a way that \emph{modular invariance} is preserved. This might seem somewhat ad hoc to define theories on a higher genus surface, but from a string theory point of view this is very natural. CFTs on higher genus surfaces describe higher order closed string scattering amplitudes. From a more stat mech point of view, a theory on a torus is simply a theory with periodic boundary conditions in both directions.
\subsection{Conformal field theory on the torus}
We will define the \emph{periods} of the lattice as linearly independent complex numbers $\omega_1,\omega_2$. The only dependence of the partition function can be on the \emph{modular} parameter $\tau=\omega_1/\omega_2$. As this is independent of the surface area and the absolute orientation of the periods.
{\bf The partition function}
In the operator formalism the partition function $Z$ and its dependence on $\tau$ stands central. We shall define the time and space direction to be the imaginary and real axis respectively. In that case:
\begin{equation}
	Z=\Tr\left(\exp(-\left\{H\Im\omega_2-iP\Re\omega_2\right\})\right),
\end{equation}
using
\begin{align}
	H &=  \frac{2\pi}{L}\left(L_0 + \overline L_0 -\frac{c}{12}\right)\\
	P &=  \frac{2\pi}{L}\left(L_0 - \overline L_0\right),
\end{align}
\footnote{{\color{red}Remark:} there is a misprint in the book: $P$ shouldn't have an $i$.} (the factor $c/12$ ensures the energy is zero in the infinite length case), we find:
\begin{equation}
	Z = \Tr(q^{L_0 - c/24}\overline q^{\overline L_0 -c/24}),
\end{equation}
where
\begin{align}
	q &= \exp(2\pi i\tau)\\
	\overline q &= \exp(-2\pi i\overline \tau).
\end{align}
We will see that the partition function will be a bilinear combination of characters.

{\bf Modular invarariance}
Suppose $\omega'_i$ define the same lattice, then necessarily:
\begin{equation}
\begin{pmatrix}
	\omega'_1\\
	\omega'_2
\end{pmatrix}
=
\begin{pmatrix}
	a & b\\
	c & d
\end{pmatrix}
\begin{pmatrix}
	\omega_1\\
	\omega_2
\end{pmatrix}.
\end{equation}
Where $\begin{pmatrix}
	a & b\\
	c & d
\end{pmatrix}\in SL(2,\mathbb{Z})$. The matrix has an integer inverse, and describes basis transformations that leave the surface area of the torus invariant.

Under a $SL(2,\mathbb{Z})$ transformation:
\begin{equation}
	\tau\mapsto\frac{a\tau + b}{c\tau +d},\qquad ad-bc=1.
\end{equation}
This leads to the modular group $SL(2,\mathbb{Z})/\mathbb{Z}_2=PSL(2,\mathbb{Z})$. Where the $\mathbb{Z}_2$ subgroup that is modded out corresponds to the symmetry $a,b,c,d\mapsto-a,-b,-c,-d$.


{\bf Generators and the fundamental domain}
The modular group is generated by:
\begin{align}
	&T:\tau\mapsto\tau +1,\qquad
	T =
	\begin{pmatrix}
		1 & 0\\
		1 & 1
	\end{pmatrix}\\
	&S:\tau\mapsto-\frac{1}{\tau},\qquad
	S =
	\begin{pmatrix}
		0 & -1\\
		1 & 0
	\end{pmatrix}.
\end{align}
And you can check ${(ST)}^3=S^2=-1$.\footnote{{\color{red}Remark:} there is a misprint in the book: $S$ is missing a minus sign and ${(ST)}^3=S^2=1$ should be $-1$.}

Geometrically, the modular group is generated by \emph{Dehn twists}: $T$ amounts to cutting the torus at fixed time and turning one of the ends $2\pi$ before gluing it back together. $U=TST$ does the same for fixed space. These are operators which are not smoothly connected to the identity.

A \emph{fundamental domain} $F$ is a part of the upper half complex plane that can be mapped to every other point of the complex plane by modular transformations. A convention for $F_0$ is:
\begin{equation}
	z\in F_0\quad\text{if}\quad
	\begin{cases}
		\Im z > 0, -\frac{1}{2}\leq \Re z\leq 0, \qquad|z|\geq 1\\
		\Im z > 0, 0< \Re z<\frac{1}{2},\qquad |z|\geq 1
	\end{cases}
\end{equation}
\subsection{The free boson on the torus}
Recall Dedekind's $\eta$ function:
\begin{equation}
	\eta(\tau) = q^{\frac{1}{24}}\prod_{n=1}^\infty (1-q^n).
\end{equation}
it can be shown that under modular transformations:
\begin{align}
	\eta(\tau+1) &= e^{i\pi/12}\eta(\tau)\\
	\eta(-1/\tau) &= \sqrt{-i\tau}\eta(\tau).
\end{align}
Without zero mode the boson partition function can be written as:
\begin{equation}
	Z_{\text{bos}}(\tau) = \frac{1}{{(\Im\tau)}^{1/2}|\eta(\tau)|^2}.
\end{equation}
This is a modular invariant combination of $\tau$ and can be obtained in the path-integral formalism by expanding the boson field in the eigenfunctions of the Laplacian and using $\zeta$-regularization.
\subsection{Free fermions on the Torus}
Fermion fields can be periodic or anti-periodic in two directions on the lattice:
\begin{align}
	\psi(z+\omega_1) &= e^{2\pi i \nu}\psi(z)\\
	\psi(z+\omega_2) &= e^{2\pi i u}\psi(z),
\end{align}
$\nu, u =0, \frac{1}{2}$, and we shall call the periodic boundary condition \emph{Ramond} and the anti-periodic boundary condition \emph{Neveu-Schwarz}. This leads to four different sectors. A set of boundary conditions is called a \emph{spin structure} for the fermion on the torus. Because $\psi$ and $\overline\psi$ decouple we can write:
\begin{equation}
	Z_{\nu, u} = |d_{\nu, u}|^2.
\end{equation}
A tedious computation leads to
\begin{align}
	d_{0,0} &= \frac{1}{\sqrt{2}} {\Tr}\, {\left(-1\right)}^F q^{L_0 - 1/48} = \frac{1}{\sqrt{2}} {\Tr}\, {(-1)}^F q^{\sum_k kb_{-k}b_k+1/24},\\
	d_{0,\frac{1}{2}} &= \frac{1}{\sqrt{2}}\Tr  q^{L_0 - 1/48} = \frac{1}{\sqrt{2}}\Tr q^{\sum_k kb_{-k}b_k+1/24},\\
	d_{\frac{1}{2},0} &= {\Tr}\, {(-1)}^F q^{L_0 - 1/48} = {\Tr}\, {(-1)}^F q^{\sum_k kb_{-k}b_k-1/48},\\
	d_{\frac{1}{2},\frac{1}{2}} &= \Tr q^{L_0 - 1/48} = \Tr q^{\sum_k k b_{-k}b_k-1/48},
\end{align}
or
\begin{align}
	d_{0,0} &= 0,\\
	d_{0,\frac{1}{2}} &= \sqrt{\frac{\theta_2(\tau)}{\eta(\tau)}},\\
	d_{\frac{1}{2},0} &= \sqrt{\frac{\theta_4(\tau)}{\eta(\tau)}},\\
	d_{\frac{1}{2},\frac{1}{2}} &= \sqrt{\frac{\theta_3(\tau)}{\eta(\tau)}}.
\end{align}
Remember the characters
\begin{equation}
	\chi_{(c,h)}(\tau) = \Tr q^{L_0 -c/24}.
\end{equation}
And the expressions for $L_0$:
\begin{align}
	L_0 &= \sum_{k>0} kb_{-k}b_k\qquad \left(k\in\mathbb{Z}+\frac{1}{2}\right),\\
	L_0 &= \sum_{k>0} kb_{-k}b_k + \frac{1}{16}\qquad \left(k\in\mathbb{Z}
	\right),
\end{align}
in the NS and R sectors.

Consider e.g.\ the NS sector. Since $L_0$ takes both half-integer and integer values, this character is the sum of at least two simple Virasoro characters. From the fact that $c=1/2$ we find that
\begin{align}
	\chi_{1,1} &= q^{-1/48}\frac{1}{2}\Tr(1+{(-1)}^F)q^{L_0}\\
	\chi_{2,1} &= q^{-1/48}\frac{1}{2}\Tr(1-{(-1)}^F)q^{L_0}.
\end{align}
Comparing with the partition function:
\begin{align}
	\chi_{1,1} &= \frac{1}{2}\left(d_{\frac{1}{2},\frac{1}{2}} + d_{\frac{1}{2},0}\right),\\
	\chi_{2,1} &= \frac{1}{2}\left(d_{\frac{1}{2},\frac{1}{2}} - d_{\frac{1}{2},0}\right).
\end{align}
Similarly for the R sector:
\begin{equation}
	\chi_{1,2} = \frac{1}{\sqrt{2}}d_{0,\frac{1}{2}}.
\end{equation}

From the modular transformation properties of the $d$'s, it follows there are two modular invariant partition functions: the one which only contains (R,R) and is zero and the one that combines the other sectors as:
\begin{align}
	Z &= Z_{\frac{1}{2},\frac{1}{2}} + Z_{0,\frac{1}{2}} + Z_{\frac{1}{2},0}\\
	&= \left|\frac{\theta_2}{\eta}\right| + \left|\frac{\theta_3}{\eta}\right| + \left|\frac{\theta_4}{\eta}\right|\\
	&= 2\left(|\chi_{1,1}|^2 + |\chi_{2,1}|^2 + |\chi_{1,2}|^2\right).
\end{align}
This is twice the partition function of the Ising model on the torus!
