\section{FRS/Categorical construction of RCFT}
The mordern lanuage (probaly not the final lanuage of course) for RCFTs is the so-called category theory. The chiral symmetry of a RCFT is described by a (semisimple) modular tensor category $\mathcal{C}$. A full CFT requires the knowledge of how to combine the chiral and anti-chiral parts. Historically modular invariance has been utilized to construct a full CFT\@. However there exist non-physical modular invariant theories. The required part is a semisimple module category $\mathcal{M}$ over $\mathcal{C}$. Physically $\mathcal{M}$ contain information of how bulk fields interact with boundaries or more generally defects. Category theory provides a unified and beautiful lanuage of how bulk, boudary and defect fields interact, which compose a full CFT\@. 

\subsection{Chiral symmetry in RCFT}

\subsection{Verlinde lines in diagonal RCFT}

\subsection{Global symmetry and orbifold RCFT with discrete torsion}
With a given CFT $\mathcal{T}$ defined on $\mathcal{M}$ with a discrete global group symmetry $G$, it is possible to define a different CFT $\tilde{T} \sim T/G$ by gauging the global symmetry $G$~\cite{Douglas:aa,Gaberdiel_2000,Chang2019}. The orbifold theory $\tilde{T}$ then has a fusion category symmetry $\mathrm{Rep}(G)$. A symmetry operator or simple object in $\mathrm{Rep}(G)$ is labeled by $G$-representation. One-dimensional symmetry operators are invitable, while higher dimensional representations correspond to non-invitable symmetries. 

We first review the global symmetry $G$ in RCFT\@. Physically each symmetry operator $g$ can be represented by a line operator $\mathcal{L}_g$. We further require that $\mathcal{L}_g$ commutes with the chiral vertext algebra in RCFT, hence $\mathcal{L}_g$ is topological, which are denoted as topological defect lines (TDLs)~\footnote{There may exist conformal line operators which are not topological.}. Two symmetry operators can be fused into a new one, in which the fusion rule follows the multiplication of group elements. When fusing three operators, there are two different ways. These two fusion orders are related by a $F$-move, characterized by $\alpha(g,h,k)\,:\, 3$-cocycles $G^3 \rightarrow U(1)$ up to coboundaries. A non-trivival 3-cocycle means the global symmetry $G$ is anomalous in which the fusion of $\mathcal{L}_g$ not associative.

When $G$ is non-anomalous, we can construct orbifold theories. We start by inserting a symmetry projector $P := \frac{1}{|G|} \sum_{g \in G} \mathcal{L}_g$ in the partition function 
\begin{equation}
	\begin{aligned}
		Z = \mathrm{Tr}\left( P\,q^{L_0} \overline{q}^{\overline{L}_0} \right).
	\end{aligned}
\end{equation}
However, this theory is usually not modular invariant, since a modular transformation can modifies the symmetry action in the time direction and generate different terms. One needs to consider the so-called twisted sectors in the Hilbert space $\mathcal{H}_h$ to construct a modular invariant theory. This twisted sector $\mathcal{H}_h$ is difind according to
\begin{equation}
	\phi(e^{2\pi i}z,e^{-2\pi i}\overline{z}) = \mathcal{L}_h \circ \phi(z,\overline{z}).
\end{equation}
Now after symmetrilization, one writes down the orbifold theory
\begin{equation}
	\begin{aligned}
		Z(q,\overline{q}) = \frac{1}{|G|}\,\sum_{[g,h]=0}\mathrm{Tr}_{\mathcal{H}_h}\left( \mathcal{L}_g \, q^{L_0} \overline{q}^{\overline{L}_0} \right).
	\end{aligned}
\end{equation}

For a given symmetry group $G$, there may exist different orbifold ways. It turns out one can use the second cohomology group $H^2(G,U(1))$ to classify orbifolds. This is known as discrete torsion in the literature. Specifically one can use $\epsilon(g,h)$ to characterize each orbifold theory
\begin{equation}
	\begin{aligned}
		Z(q,\overline{q}) = \frac{1}{|G|} \, \sum_{[g,h]=0} \epsilon(g,h) \, \mathrm{Tr}_{\mathcal{H}_h}\left( \mathcal{L}_g \, q^{L_0} \overline{q}^{\overline{L}_0} \right),
	\end{aligned}
\end{equation}
where $\epsilon(g,h) := \nu(g,h)/\nu(h,g)$ with $\nu(g_1,g_2)$ being two cocycles. The symmetry projector becomes $P_h := \frac{1}{|G|}\sum_g \epsilon(g,h)\,\mathcal{L}_g$. 

\subsection{Duality TDL and Self-dual orbifold}

\subsection{Topological defects in Ising CFT}
Besides the closed manifolds, CFTs can also be defined on manifols with boundaries, or defects in more general setting. A boundary can be understood a special defect between a CFT and a trivial CFT with a central charge $c=0$. On the other hand a defect between CFT1 and CFT2 can be regarded as a boundary of stacking the two CFTs using the folding trick. It is natural to require the conformal symmetry still holds at boundaries and defects. For a boundary state $\vert a \rangle$ this means 
\begin{equation}
	\left ( L_n - \overline{L}_{-n} \right ) \vert a \rangle = 0.
\end{equation}
For a defect $O_\alpha$ it requires
\begin{equation}
	\left[ L_n - \overline{L}_{-n},\, O_\alpha \right] = 0.
\end{equation}
Usually we require stronger constraint for the conformal defects 
\begin{equation}
	\begin{aligned}
		\left[ L_n,\, O_\alpha \right] = 0 \\
		\left[ \overline{L}_{n},\, O_\alpha \right] = 0
	\end{aligned}
\end{equation}
which are now called topological defects. They correspond to the topological aspects of CFT and serve the role of intertwiners between conformal boundaries. Since the topological defects commute with Virasoro generators, they can be viewed as genearlized symmetries. 

Let's start with a simple example --- the Ising CFT $\mathcal{M}(4,3)$. There are three conformal boundaries/topological defects in Ising CFT which are labeled by primaries. Two defects can be fused into other defects when put close to each other. As a result their fusion forms a fusion algebra. In Ising CFT, the fusion algebra of defects is the same as that of primaries. This is true for all A-type minimal models. 

\subsection{Boundary states}

\subsection{Potts CFT}
Let's summarise the above contents using a simple, but non-trivial example. This is the minimal model $M(6,5)$ or the Potts CFT with a central charge $c=\frac{5}{6}$. The chiral operator contents for the $M(6,5)$ model are listed in Table.~\ref{kacm65}. There are $10$ chiral primaries in the minimal model $M(6,5)$. One of them is a spin-$3$ current operator $\Phi_{4,1}$ (or $\Phi_{1,5}$), which is the so-called simple current. 

\subsubsection{Modular invariant partition function}
It turns out one can construct different modular invariant partition functions
\begin{equation}
	Z = \sum_{i,j} \chi_i(q) \, M_{ij}\, \chi_j(\bar{q})
\end{equation}
where $M_{ij}$ contians only non-negative integers as its elements. One modular invariant belongs to the $A$ diagonal series and $M_{ij}$ is nothing but a 10-dimensional unit matrix. 

\begin{table}
	\begin{center}
	\renewcommand{\arraystretch}{1.6}
	\begin{tabular}{|c | c | c | c | c|} 
		\hline 
		Kac label & $\Delta$ & label & meaning & Lattice operator\\
		\hline
		(1,1) & 0 & $\mathcal{I}$ &identity& \\
		\hline
		(4,1) & 3 & $\mathcal{I}'$ & spin-3 current & \\
		\hline
		(2,1) & $\frac{2}{5}$  & $\mathcal{\epsilon}$ &thermal operator & $Z_i Z_{i+1}^\dagger - \frac{1}{2}(X_i+X_{i+1})+h.c.\sim \Phi_{\epsilon,\bar{\epsilon}}$ \\
		\hline
		(3,1) & $\frac{7}{5}$ & $\mathcal{\epsilon}'$& & \\
		\hline
		(3,3) & $\frac{1}{15}$  & $\mathcal{\sigma}$ & spin operator& $Z_i \sim \Phi_{\sigma,\bar{\sigma}}(z,\bar{z})$\\
		\hline
		(4,3) & $\frac{2}{3}$ & $\psi$ & &\\
		\hline \hline
		(4,2) & $\frac{13}{8}$ & & &\\
		\hline
		(2,2) & $\frac{1}{40}$ & & &\\
		\hline
		(3,2) & $\frac{21}{40}$ & & &\\
		\hline 
	\end{tabular}
	\caption{Chiral primaries of $M(6,5)$ minimal model. The primaries above the double line are those shown in the Potts model.}
		\label{kacm65}
	\end{center}
\end{table}

\begin{table}
	\begin{center}
	\renewcommand{\arraystretch}{1.6}
	\begin{tabular}{|c | c | c | c | c|} 
		\hline 
		Kac label & $\Delta$ & label & meaning & Lattice operator\\
		\hline
		(1,1)+(4,1) & 0 & $\mathcal{I}$ &identity& \\
		\hline
		(2,1)+(3,1) & $\frac{2}{5}$  & $\mathcal{\epsilon}$ &thermal operator & $Z_i Z_{i+1}^\dagger - \frac{1}{2}(X_i+X_{i+1})+h.c.\sim \Phi_{\epsilon,\bar{\epsilon}}$ \\
		\hline
		(3,3) & $\frac{1}{15}$  & $\mathcal{\sigma}$ & spin operator ($Z_3$ charge $q=1$) & $Z_i \sim \Phi_{\sigma,\bar{\sigma}}(z,\bar{z})$\\
		\hline
		(3,3) & $\frac{1}{15}$  & $\mathcal{\sigma}^\dagger$ & spin operator ($Z_3$ charge $q=2$)& $Z_i \sim \Phi_{\sigma,\bar{\sigma}}(z,\bar{z})$\\
		\hline
		(4,3) & $\frac{2}{3}$ & $\psi$& $Z_3$ charge $q=1$ & \\
		\hline 
		(4,3) & $\frac{2}{3}$ & $\psi^\dagger$&$Z_3$ charge $q=2$ & \\
		\hline 
	\end{tabular}
	\caption{Chiral primaries of the parafermion CFT - the critical three-state Potts model.}
		\label{potts_kacm65}`'
	\end{center}
\end{table}

The other one uses the simple current to form block characters and belongs to a non-diagonal theory. One first needs to notice that the Potts model has an internal $Z_3$ symmetry. The $\Phi_{3,3}$ and $\Phi_{4,3}$ primaries splits into two spaces with different $Z_3$ charges. Under the action of the simple current (here we denote the spin-3 current operator as $J(z) := \Phi_{4,1}(z)$),
\begin{equation}
	\begin{aligned}
		& J \times I \sim J\\
		& J \times J \sim I\\
		& J \times \epsilon \sim \epsilon'\\
		& J \times \epsilon' \sim \epsilon\\
		& J \times \sigma_i \sim \sigma_i\\
		& J \times \psi_i \sim \psi_i
	\end{aligned}
\end{equation}
There are two 2-orbits and two 1-orbits. Using the block-characters $C_i$, the partition function reads
\begin{equation}
	\label{eq_z_potts}
	Z = \vert C_I(\tau) \vert^2 + \left\vert C_\epsilon(\tau) \right\vert^2 + \left\vert C_{\sigma_1}(\tau) \right\vert^2 + \left\vert C_{\sigma_2}(\tau) \right\vert^2 + \left\vert C_{\psi_1}(\tau) \right\vert^2 + \left\vert C_{\psi_2}(\tau) \right\vert^2
\end{equation}
Another way to look at this follows similar ideas in the $su(2)_k$ CFTs, where the Virasoro algebra is extended by a larger one. Here Potts model is extended by $W_3$ algebra. The $\Phi_{3,3}$ and $\Phi_{4,3}$ are not faithful representation of the $W_3$ algebra. Each one splits into two $W_3$-primaries. They contian different $Z_3$ charges.  From this viewpoint, one should use $W_3$-primaries rather than the Virasoro-primaries in the modular invariant, see Table.~\ref{potts_kacm65}, which gives the same modular invariant Eq.~\ref{eq_z_potts}.