\section{Appendix}
\subsection{Central extensions of Lie algebras}
In this section $\mathfrak{g},\mathfrak{h},\cdots $ denote (possibly infinite) Lie algebras over some field $\mathbb{K} = \mathbb{R}, \mathbb{C}$. This section is mainly based on Wikipedia and~\cite{Schottenloher}.


\subsubsection*{Extensions}
{\bf Definition:} A \emph{Lie algebra extension} is a short exact sequence of Lie algebras:
\begin{equation}
	\mathfrak{h}\overset{\iota}{\rightarrow}\mathfrak{e}\overset{\pi}{\rightarrow}\mathfrak{g}.
\end{equation}
One calls $\mathfrak{e}$ an extension of $\mathfrak{g}$ by $\mathfrak{h}$. By exactness of the sequence one has $\mathfrak{g}\cong\mathfrak{e}\slash\Ima\iota$.\\

{\bf Definition:} A \emph{central extension} is an extension $\mathfrak{e}$ of $\mathfrak{g}$ by $\mathfrak{h}$, such that $\Ima \iota$ is contained in the center of $\mathfrak{e}$, $\iota(\mathfrak{h})\subseteq Z(\mathfrak{e})$.\\

Notice that for a central extension $\mathfrak{h}$ is necessarily abelian. We now introduce a notion of trivial central extensions as follows:

{\bf Definition:} A Lie algebra extension
\begin{equation}
	\mathfrak{h}\overset{\iota}{\rightarrow}\mathfrak{e}\overset{\pi}{\rightarrow}\mathfrak{g}
\end{equation}
\emph{splits} if there exists a Lie algebra morphism $\beta: \mathfrak{g}\mapsto\mathfrak{e}$ such that $\pi\circ\beta = \text{id}_{\mathfrak{e}}$. $\beta$ is called a splitting map.\\

A central extension
\begin{equation}
	\mathfrak{h}\overset{\iota}{\rightarrow}\mathfrak{e}\overset{\pi}{\rightarrow}\mathfrak{g}.
\end{equation}
that splits is trivial in the sense that it is equivalent\footnote{To do: introduce the notion of equivalent extensions.} to one where $\mathfrak{e}\cong\mathfrak{g}\oplus\mathfrak{h}$.\\

Let us now consider a central extension and a map (not necesserily a Lie algebra homomorphism) $\beta:\mathfrak{g}\rightarrow\mathfrak{e}$ such that $\pi\circ\beta = \text{id}_{\mathfrak{e}}$. From this map construct
$\Theta: \mathfrak{g}\times\mathfrak{g}\rightarrow\mathfrak{h}$ as follows:
\begin{equation}
	\Theta(x,y) := \left[\Theta(x),\Theta(y)\right] - \Theta\left([x,y]\right).
\end{equation}
This map is:
\begin{enumerate}
	\item \label{prop:antisym} Antisymmetric.
	\item \label{prop:bilinear} Bilinear.
	\item \label{prop:Jacobi} Satisfies $\Theta(x,[y,z]) + \Theta(y,[z,x]) + \Theta(z,[x,y]) = 0$.
\end{enumerate}
Given $\Theta$ one can now show that there is an isomorphism between the vector spaces $\mathfrak{e}\cong\mathfrak{g}\oplus\mathfrak{h}$ that is given by:
\begin{equation}
	\Psi:\mathfrak{g}\oplus\mathfrak{h}\mapsto\mathfrak{e}:(x,y)\mapsto\beta(x) + y.
\end{equation}
A Lie bracket on $\mathfrak{g}\oplus\mathfrak{h}$ is given by:
\begin{equation}
	[x\oplus z, y\oplus z']_{\mathfrak{e}} := [x,y]_\mathfrak{g} + \Theta(x,y).
\end{equation}

{\bf Lemma:} In the above construction $\beta$ is a splitting map if and only if
\begin{equation}
	\Theta(x,y) = \mu([x,y]),
\end{equation}
for some $\mu\in\text{Hom}(\mathfrak{g},\mathfrak{h})$.\\

Now comes the classification of central extensions of Lie algebras:\\

{\bf Theorem:} Every central extension comes from a map $\Theta$ that satisfies the above properties (\ref{prop:antisym}-\ref{prop:Jacobi}). Conversely, every central extension gives rise to a map $\Theta$ that satisfies the above properties (\ref{prop:antisym}-\ref{prop:Jacobi}).

\subsubsection*{Lie algebra cohomology}
The classification of Lie algebra extensions is very satisfying. It smells a lot like a cohomological classification. Indeed, the extensions are classified by functions depending on two variables satisfying the condition (\ref{prop:Jacobi}) that is exactly the one needed to fulfill the Jacobi identity of the central extension. Moreover, the central extension is trivial if the \emph{2-cocycle} $\Theta$ is trivial in the following sense: $\Theta(x,y) = \mu([x,y])$. This is reminiscent of considering 2-cocycles to be trivial if they are equal to a coboundary. Let us put this on a bit more rigorous footing.\\

{\bf Definitions:}
\begin{enumerate}
	\item $Z^2(\mathfrak{g},\mathfrak{h}) = \left\{\Theta\in\Lambda^2(\mathfrak{g},\mathfrak{h})|\Theta:(\ref{prop:Jacobi})\right\}$.
	\item $B^2(\mathfrak{g},\mathfrak{h})=\left\{\Theta:\mathfrak{g}\times\mathfrak{g}\mapsto\mathfrak{h}|\exists\mu\in\text{Hom}(\mathfrak{g},\mathfrak{h}):\Theta(-,-)=\mu([-,-])\right\}$.
	\item $H^2(\mathfrak{g},\mathfrak{h}):=Z^2(\mathfrak{g},\mathfrak{h})/B^2(\mathfrak{g},\mathfrak{h})$.
\end{enumerate}
$H^2$ is of course called the second cohomology group. We thus obtain the following reformulation of the classification of central extensions:\\

{\bf Theorem:} The equivalence classes of central extensions
\begin{equation}
	\mathfrak{h}\overset{\iota}{\rightarrow}\mathfrak{e}\overset{\pi}{\rightarrow}\mathfrak{g}
\end{equation}
are in one-to-one correspondence with the elements of $H^2(\mathfrak{g},\mathfrak{h})$.\\

For completeness, let us introduce a notion of cochain complexes for Lie algebras. A cochain $f$ is a alternating multilinear map $f$:
\begin{equation}
	f:\Lambda^n \mathfrak{g} \mapsto \mathfrak{h}.
\end{equation}
Here, $\mathfrak{h}$ is considered a $\mathfrak{g}$-module or -representation.

The differential of an $n$-cochain is given by
\begin{equation}
\begin{aligned}
	(d f)\left(x_1, \ldots, x_{n+1}\right) =
	&\sum_i     (-1)^{i+1}x_i\, f\left(x_1, \ldots, \hat x_i, \ldots, x_{n+1}\right) + \\
	&\sum_{i<j} (-1)^{i+j}      f\left(\left[x_i, x_j\right], x_1, \ldots, \hat x_i, \ldots, \hat x_j, \ldots, x_{n+1}\right)\, ,
\end{aligned}
\end{equation}

so for example, with trivial action we obtain
\begin{equation}
	(df)(x_1,x_2) = f([x_1,x_2]),
\end{equation}
and
\begin{equation}
\begin{aligned}
	(df)(x_1,x_2,x_3) &= -f([x_1,x_2],x_3) + f([x_1,x_3],x_2) - f([x_2,x_3],x_1)\\
	&= -f([x_1,x_2],x_3) - f([x_3,x_1],x_2) - f([x_2,x_3],x_1)\\
	&= f(x_3,[x_1,x_2]) + f(x_2,[x_3,x_1]) + f(x_1,[x_2,x_3]).
\end{aligned}
\end{equation}
So clearly, $Z^2(\mathfrak{g},\mathfrak{h})$ defined above is the group of 2-cocycles satisfying $d\Theta=0$ and $B^2(\mathfrak{g},\mathfrak{h})$ the set of coboundaries: $\Theta = d\mu$.

