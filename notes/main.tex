% =========================================================================
% SciPost LaTeX template
% Version 2021-08
%
% Submissions to SciPost Journals should make use of this template.
%
% INSTRUCTIONS: simply look for the `TODO:' tokens and adapt your file.
%
% You can also make use of our empty "skeleton" templates for each Journal,
% e.g. SciPostPhys_skeleton.tex
% =========================================================================


% TODO: uncomment ONE of the class declarations below

% Class declaration format: \documentclass[submission, {DOI label of journal}]{SciPost}
% where the DOI label of the journal should be one of:
% Phys          (for SciPost Physics)
% PhysCore      (for SciPost Physics Core)
% PhysLectNotes (for SciPost Physics Lecture Notes)
% PhysProc      (for SciPost Physics Proceedings -> !! Please use the conference-specific template which you will find on our website !!
% PhysCodeb     (for SciPost Physics Codebases)
% Astro         (for SciPost Astronomy)
% Bio           (for SciPost Biology)
% Chem          (for SciPost Chemistry)
% CompSci       (for SciPost Computer Science)
% Math          (for SciPost Mathematics)


%% PHYSICS:
% If you are submitting a paper to SciPost Physics: uncomment next line
%\documentclass[submission, Phys]{SciPost}
% If you are submitting a paper to SciPost Physics Core: uncomment next line
%\documentclass[submission, PhysCore]{SciPost}
% If you are submitting a paper to SciPost Physics Lecture Notes: uncomment next line
\documentclass[submission, PhysLectNotes]{SciPost}
% If you are submitting a paper to SciPost Physics Proceedings: uncomment next line
%\documentclass[submission, PhysProc]{SciPost}
% If you are submitting a paper to SciPost Physics Codebases: uncomment next line
%\documentclass[submission, PhysCodeb]{SciPost}



% Prevent all line breaks in inline equations.
\binoppenalty=10000
\relpenalty=10000

\hypersetup{
    colorlinks,
    linkcolor={red!50!black},
    citecolor={blue!50!black},
    urlcolor={blue!80!black}
}

\usepackage[bitstream-charter]{mathdesign}
\usepackage{amsmath}
\urlstyle{sf}

% Fix \cal and \mathcal characters look (so it's not the same as \mathscr)
\DeclareSymbolFont{usualmathcal}{OMS}{cmsy}{m}{n}
\DeclareSymbolFontAlphabet{\mathcal}{usualmathcal}

\DeclareMathOperator{\Ima}{Im}

%----------hrz macros
\newcommand{\ExOp}[1]{\left\langle#1\right\rangle}

\begin{document}

% TODO: write your article's title here.
% The article title is centered, Large boldface, and should fit in two lines
\begin{center}{\Large \textbf{
Notes for  the reading club\\
}}\end{center}

% TODO: write the author list here. Use first name (+ other initials) + surname format.
% Separate subsequent authors by a comma, omit comma and use "and" for the last author.
% Mark the corresponding author with a superscript star.
\begin{center}
Reading Club%\textsuperscript{1},
%Aah B. Cee\textsuperscript{2} and
%Gee K. See\textsuperscript{3$\star$}
\end{center}

% TODO: write all affiliations here.
% Format: institute, city, country
\begin{center}
%{\bf 1} Affiliation1
%\\
%{\bf 2} Affiliation2
%\\
%{\bf 3} Affiliation2
%\\
% TODO: provide email address of corresponding author
%${}^\star$ {\small \sf CorrespondingAuthor@email.address}
\end{center}

\begin{center}
\today
\end{center}

% For convenience during refereeing (optional),
% you can turn on line numbers by uncommenting the next line:
%\linenumbers
% You should run LaTeX twice in order for the line numbers to appear.

\section*{Abstract}
{\bf
The Yellow Book Notes. It is good to write notes!
}


% TODO: include a table of contents (optional)
% Guideline: if your paper is longer that 6 pages, include a TOC
% To remove the TOC, simply cut the following block
\vspace{10pt}
\noindent\rule{\textwidth}{1pt}
\tableofcontents\thispagestyle{fancy}
\noindent\rule{\textwidth}{1pt}
\vspace{10pt}


\section{Preliminary}
\subsection{Conventions}
{\it Metric tensor and Coordinate.--}  The metric tensor in Minkowski and Eclidean space-time is defined as
\begin{equation}
    \eta = \begin{Bmatrix}
        +1 &   &\\
           & -1& \\
           &   & ...
    \end{Bmatrix}
\end{equation}
and
\begin{equation}
    \eta = \begin{Bmatrix}
        +1 &   &\\
           & +1& \\
           &   & ...
    \end{Bmatrix}
\end{equation}
respectively, where the first index is the time. In the Yellow Book, without specifications, we are working in Eclidean space. The coordinate is defined as $x^\mu = \{t, \mathop{x}\limits^\rightarrow\}$. So that the norm of a vector in Minkowski space-time is $x^\mu x_\mu = t^2 - r^2$.

{\it $\gamma$ matrices.--}   The $\gamma$ matrices follow the Clifford algebra
\begin{equation}
    \{\gamma^a,\gamma^b\} = 2\eta^{ab}.
\end{equation}
In Minkowski space time, the $\gamma$ matrices can be chosen as
\begin{eqnarray}
    \gamma^0 &=& \sigma^x \nonumber \\
    \gamma^1 &=&  -i\sigma^y,
\end{eqnarray}
while in Elidean space, they can be chosen as
\begin{eqnarray}
    \gamma^0 &=& \sigma^x \nonumber \\
    \gamma^1 &=& \sigma^y.
\end{eqnarray}

{\it Complex Coordinate.--}  In $2d$ CFT, complex coordinate is powerful as one can utilize the beautiful analytic properties of holomorphic functions. It is defined as 
\begin{equation}
\begin{aligned}
z &= z^0 + i z^1 \\
\bar{z} &= z^0 - i z^1.
\end{aligned}
\end{equation}  
Note that here $z$ and $\bar{z}$ denote coordinates with upper index.

\subsection{Free fermions}
In Minkowski space time, the Lagrange density for the free fermion reads
\begin{equation}
    \mathcal{L} = \frac{g}{2}\left(\psi^1\ i(\partial t + \partial x)\ \psi^1 + \psi^2\ i(\partial t - \partial x)\ \psi^2 \right).
\end{equation}
In terms of $\psi = (\psi^1,\psi^2)$, one can write the theory as
\begin{eqnarray}
    \mathcal{L} &=& \frac{g}{2} \left ( \ \psi^\dagger i\partial_t \psi + \psi^\dagger \ \sigma^z \ i\partial_x \psi \ \right) \nonumber \\
    &=& \frac{g}{2} \left ( \ \psi^\dagger \sigma^x\sigma^x i\partial_t \psi + \psi^\dagger \ -i\sigma^x\sigma^y \ i\partial_x \psi \ \right) \nonumber\\
    &=& \frac{g}{2} \psi^\dagger \sigma^x \left(\sigma^x i \partial_t - i\sigma^y i\partial_x \right) \psi \nonumber \\
    &=& \frac{g}{2} \psi^\dagger \gamma^0 i\gamma^\mu \partial_\mu \psi
\end{eqnarray}
where we used
\begin{equation}
    \gamma^0 = \sigma^x \quad \gamma^1 = -i\sigma^y
\end{equation}

{\it Wick rotation}
It is usually more convenient to work in Euclidean space rather than Minkowski space time. Upon doing the Wick rotation, the action changes as
\begin{equation}
    i\ S_M \rightarrow -\ S_E.
\end{equation}
Specifically,
\begin{eqnarray}
    i\ S[\psi] &=& i\ \int dxdt \ \frac{g}{2} \psi^\dagger \gamma^0 i\gamma^\mu \partial_\mu \psi \nonumber \\
    &=& i^2 \ \int dxdt \ \frac{g}{2} \psi^\dagger \partial_t \psi + i^2 \ \int dxdt \ \frac{g}{2} \psi^\dagger \sigma^x (-i) \sigma^y \partial_x \psi \nonumber \\
    &=& -\int dxd\tau \ \frac{g}{2} \psi^\dagger \partial_\tau \psi - \int dxd(-i t) \ \frac{g}{2} \psi^\dagger \sigma^x \sigma^y \partial_x \psi \nonumber \\
    &=&  -\int dxd\tau \ \frac{g}{2} \psi^\dagger \sigma^x\sigma^x\partial_\tau \psi - \int dxd\tau \ \frac{g}{2} \psi^\dagger \sigma^x \sigma^y \partial_x \psi \nonumber \\
    &=&  -\int dxd\tau \ \frac{g}{2} \psi^\dagger \gamma^0_E\gamma^\mu_E\partial_\mu \psi
\end{eqnarray}
where $\tau = -it$. The Eulidean space action can be written as
\begin{equation}
    S_E = \int d^2x \frac{g}{2} \ \psi^\dagger \gamma^0_E\gamma^\mu_E\partial_\mu \psi
\end{equation}

{\it $1+1d$ free fermions: Legendre transformation}
A lattice version free fermion theory Eq.~$2.38$ reads
\begin{equation}
    \mathcal{L} = \frac{i}{2} \sum_n \left( \psi_n \dot{\psi_n} + \psi_n \psi_{n+1}\right).
\end{equation}
The canonical momentum corresponding to $\psi_n$ is
\begin{equation}
    \pi_n = \frac{\partial \mathcal{L}}{\partial \dot{\psi_n}} = -\frac{i}{2}\psi_n.
\end{equation}
So that the Hamiltonian is
\begin{eqnarray}
    \mathcal{H} &=& \sum_n \pi_n \dot{\psi_n} - \mathcal{L} \nonumber\\
      &=& -\frac{i}{2}\sum_n\psi_n \dot{\psi_n} - \frac{i}{2} \sum_n \left( \psi_n \dot{\psi_n} + \psi_n \psi_{n+1}\right) \\
      &=& -i \sum_n\psi_n \dot{\psi_n} - \frac{i}{2} \sum_n \psi_n \psi_{n+1} \nonumber.
\end{eqnarray}
While it shoud be
\begin{equation}
    \mathcal{H} = -\frac{i}{2}\sum_n\psi_n \psi_{n+1}.
\end{equation}
If we'd like to keep defining the derivative of Grassmann number according to the order of left-to-right, we need to define the Hamiltonian as
\begin{equation}
    \mathcal{H} = \sum_n \dot{\psi_n} \pi_n - \mathcal{L} =  - \frac{i}{2} \sum_n \psi_n \psi_{n+1}
\end{equation}

\subsection{Free boson}
The action for the free boson in the Minkoswki space time reads
\begin{equation}
    S = \frac{1}{2}g\int \ dxdt \ \partial_\mu \phi \partial^\mu \phi,
\end{equation}
where $\phi$ is a real scalar field. After Wick rotation $\tau = it$, it becomes
\begin{eqnarray}
    i\ S &=& \frac{i}{2}g\int \ dxdt \ \partial_t \phi \partial_t \phi - \frac{i}{2}g\int \ dxdt \ \partial_x \phi \partial_x \phi \nonumber \\
    &=& -\frac{1}{2}g\int \ dxd\tau \ \partial_\tau \phi \partial_\tau \phi - \frac{1}{2}g\int \ dxdit \ \partial_x \phi \partial_x \phi \nonumber \\
    &=& -\frac{1}{2}g\int \ dxd\tau \ \partial_\mu \phi \partial^\mu \phi
\end{eqnarray}
The Euclidean action reads
\begin{equation}
    S_E = \frac{1}{2}g\int \ d^2x \ \partial_\mu \phi \partial^\mu \phi.
\end{equation}
The two point correlation up to a constant term is
\begin{equation}
    \langle \phi(x) \phi(y) \rangle = -\frac{1}{2\pi g} \mathrm{ln} (\rho).
\end{equation}

\subsection{Symmetries at the classical level}
The action becomes different after a coordinate transformation. We say it has a symmetry if it remains unchanged and a Noether current can be derived from the symmetry. The coordinate transformation is denoted as
\begin{equation}
    x'^\mu = x^\mu + \omega_a \frac{\delta x^\mu}{\delta \omega_a}
\end{equation}
and the field changes according to
\begin{equation}
    \phi'(x') = \phi(x) + \omega_a \frac{\delta F}{\delta \omega_a}(x)
\end{equation}
where $\omega_a$ is a constant and small parameter.

By definition, the change of the action $\delta S$ disappears for a symmetric transformation. We can get nothing new from this. If we allow $\omega_a$ to be arbitrary, the leading contribution to $\delta S$ becomes
\begin{equation}
    \delta S = -\int d^2x j^\mu \partial_\mu \omega_a,
\end{equation}
where we introduced the the current $j^\mu$. We assume it decreases fast when approaching infinite. So that one obtains
\begin{equation}
    \delta S = \int d^2x\ \partial_\mu j^\mu \ \omega_a.
\end{equation}
This equations holds for all the field configurations. If we require the field configuration to be the one obeying the equation, the action should be invariant for arbitrary coordinate transformation and one finds the conservation of $j^\mu$
\begin{equation}
    \partial_\mu j^\mu = 0.
\end{equation}

{\it Energy-momentum tensor}
The canonical energy-momentum tensor is defined to be the Noether current of the translation transformation
\begin{eqnarray}
    x'^\mu &=& x^\mu + \epsilon^\nu \delta^\mu_\nu \\
    T^{\mu\nu} &=& -\eta^{\mu\nu} L + \frac{\partial L}{\partial(\partial_\mu \phi)}\partial_\nu \phi.
\end{eqnarray}
This definition of $T^{\mu\nu}$ is not guaranteed to be symmetric between the two indices (The requirement of a symmetric $T^{\mu\nu}$ will be clear later).

Another definition that makes the energy-momentum tensor symmetric follows. In the coordinate transformation, if we also consider the variance of the metric tensor (which means the theory is coupled with the dynamical backgroud)
\begin{equation}
\delta g_{\mu\nu} = -\partial_\mu\epsilon_\nu -\partial_\nu\epsilon_\mu
\end{equation}
the action remains invariant since this is nothing but a reparametrization of the theory (general coordinate covariance). So that one finds
\begin{equation}
    \delta S = 0 = -\frac{1}{2} \int d^dx \ \left(\partial_\mu\epsilon_\nu + \partial_\nu\epsilon_\mu\right) \left(T^{\mu\nu} +2\frac{\delta S}{\delta g_{\mu\nu}}\right).
\end{equation}
So that one can define the energy-momentum tensor as
\begin{equation}
    T^{\mu\nu} = -2\frac{\delta S}{\delta g_{\mu\nu}}
\end{equation}
up to a surface term.

Another way to make the energy-momentum tensor symmetric is add a surface term to the canonical one. One can show that with rotation symmetry, such a term can be constructed to make $T^{\mu\nu}$ symmetric.

\subsection{Symmetry at the quantum level}
All the field configurations contribute to the quantum theory, so that one has no Noether current in general. Still the symmetry has constraints to the quantum theory. For the $n-$point correlation functions, one has
\begin{eqnarray}
\langle \phi(x'_1)...\phi(x'_n) \rangle &=& \frac{1}{Z}\int [D\phi]\ \phi(x'_1)...\phi(x'_n) \ e^{-S[\phi]} \\
&=& \frac{1}{Z}\int [D\phi']\ \phi'(x'_1)...\phi'(x'_n) \ e^{-S'[\phi']} \\
&=& \frac{1}{Z}\int [D\phi]\ F(\phi(x_1))...F(\phi(x_n)) \ e^{-S[\phi]} \\
&=& \langle\ F(\phi(x_1))...F(\phi(x_n)) \ \rangle
\end{eqnarray}
in which we assumed the functional integral measure does not change and the coordinate transformation is a rigid one ($\omega_a$ is a constant).

{\it Ward identity}
As stated above there is no conserved current at the quantum level. The infinitesimal coordinate transformation at the quantum level results in the so-called Ward identity.

We denote the change of fields as
\begin{equation}
    \phi'(x) = \phi(x) -i\omega_a\ G_a\ \phi(x).
\end{equation}
The infinitesimal coordinate transformation ($\omega_a$ now is arbitrary) changes the correlation as (We only consider the first order perturbation contribution)
\begin{eqnarray}
\langle \phi'(x_1)... \phi'(x_n)\rangle &=& \langle \phi(x_1)... \phi(x_n)\rangle \\
&=& \frac{1}{Z} \int [D\phi'] (X+\delta X) e^{-S[\phi] - \int d^dx\partial_\mu j^\mu \omega_a} \\
&=& \frac{1}{Z} \int [D\phi] (X+\delta X) e^{-S[\phi] - \int d^dx\partial_\mu j^\mu \omega_a} \\
&=& \langle X \rangle - \int [D\phi] \int d^dx\ X \partial_\mu j^\mu \omega_a e^{-S[\phi]} - \int [D\phi] \delta X  e^{-S[\phi]}
\end{eqnarray}
so that one finds
\begin{equation}
    \langle\delta X\rangle = \int d^dx \ \partial_\mu\langle j^\mu \ X\rangle \omega_a(x).
\end{equation}
As
\begin{eqnarray}
\delta X &=& -i \sum_i \phi(x_1)...G_a \phi(x_i)...\phi(x_n)\omega_a(x_i) \\
&=& -i \int d^dx \sum_i \phi(x_1)...G_a \phi(x_i)...\phi(x_n)\delta(x-x_i)\omega_a(x)
\end{eqnarray}
Since $\omega_a$ is arbitrary, one obtains the Ward identity
\begin{equation}
    \partial_\mu\langle j^\mu \ X\rangle = -i \sum_i \delta(x-x_i)\ \langle \phi(x_1)...G_a \phi(x_i)...\phi(x_n).
\end{equation}
So that for each symmetry, there exists a Ward identity, i.e., a constraint to the correlation function. With enough symmetries, one can get all the information of the correlation functions.

\subsection{Renormalization group}
{\it Dimensional analysis and renormalizability of QFT}
Let's start with the canonical dimension of fields and couplings in the action,
\begin{equation}
    S = \int d^dx \ \mathcal{L}(\phi, \lambda).
\end{equation}
Since the action is dimensionless, every term in $\mathcal{L}$ has an energy scaling dimension of
\begin{equation}
    \Delta(\mathcal{L}) = [\mathcal{L}] = \omega^d
\end{equation}
which determines the canonical dimension fields and couplings. The renormalizability of a QFT is directly obtained from the energy dimension of Feynman diagrams,
\begin{equation}
    \mathcal{D} = d - E_{\phi} \Delta (\phi) - \Delta (\lambda_i)
\end{equation}
where $E_{\phi}$ is the number of external fields and $\lambda_i$ the couplings in the theory. A nice discussion about renormalizability can be found online (https://web2.ph.utexas.edu/~vadim/
Classes/2022f/notes.html).

Super-renormalizable theories have only couplings with positive dimensions. For such theories, there are finite Feynman diagrams become divergent in the perturbation calculation. Renormalizable theories have couplings with non-negative dimensions, in which a finite number of couplings have zero dimensions. There exists infinite number of divergent Feynman diagrams, but the number of divergent amplitudes is finite. If there is at least one coupling with a negative dimension, the theory is non-renormalizable.

{\it Wilson-Kadanoff RG scheme}
The renormalization group (RG) builds up the modern understanding of QFT, which is regarded as an {\it effective field theory}. In the history, many different RG schemes have been developed, which are suitable for very different theories. Most of them are realized in a perturbation way around a known RG fixed point. Here we briefly recall the most popular one, i.e. the Wilson-Kadanoff RG scheme.

In this scheme, a momentum cutoff $\boldsymbol{k}<\Lambda$ is introduced. One first divides modes into fast $\Lambda/s<\boldsymbol{k}<\Lambda$ and slow $\boldsymbol{k}<\Lambda/s$ parts $\phi = \phi_f +\phi_s$. The fast modes are integrated out to result in a new theory
\begin{equation}
    e^{-S'(\phi)_{\Lambda/s}} = \int \mathrm{D}\phi_{\Lambda/s<\boldsymbol{k}<\Lambda} \ e^{-S_{\Lambda}(\phi)}
\end{equation}
with a smaller cutoff $\Lambda/s$. Generally, the action can be divided into three parts
\begin{equation}
S = S_f(\phi_f) + S_s(\phi_s) + S_c(\phi_f,\phi_s).
\end{equation}
The new theory thus can be written as
\begin{equation}
\begin{aligned}
    e^{-S'(\phi_s)_{\Lambda/s}} &= \int D\phi_f e^{-S_f - S_s - S_c} \\
    &= e^{-S_s} Z_f \ \frac{\int \mathrm{D}\phi_f \ e^{-S_f} e^{-S_c}} {Z_f} \\
    &= e^{-S_s}\ Z_f \ \langle e^{-S_c} \rangle_f
\end{aligned}
\end{equation}
where $ Z_f = \int \mathrm{D}\phi_f \ e^{-\phi_f}$ is a constant and can be neglected (Note that it does contribute to the total free energy). The new action thus is
\begin{equation}
\begin{aligned}
S(\phi_s)_{\Lambda/s} &= -\mathrm{log} \left( \int \mathrm{D}\phi_{\Lambda/s<\boldsymbol{k}<\Lambda} \ e^{-S_{\Lambda}(\phi)} \right) \\
&= S_s - \mathrm{log} \left(\langle e^{-S_c} \rangle_f\right)
\end{aligned}
\end{equation}
Usually one can not integral out high energy modes exactly, hence cumulant perturbations based on Feynmann diagramm have to be adopted.

This theory can not be compared with the original one, since they have different cutoffs. Another rescaling step
\begin{equation}
    \boldsymbol{k} \rightarrow s\ \boldsymbol{k}
\end{equation}
is required to restore the cutoff or energy scale. Since the field operators dependend on length scales, they also need to be rescaled
\begin{equation}
     \phi \rightarrow s^{\Delta_\phi}\ \phi
\end{equation}
Now one obtains a new theory $S(\phi,\lambda)_\Lambda$ at the same cutoff but with different parameters, in which we assumed the theory $S(\phi,\lambda)$ remains the same structure.

Keep doing such RG procedures, one can find how the parameters $\lambda_i(s)$ flow in the parameter space along with the RG time $s$. These RG transformations of the parameters form a semi-group structure. In the whole parameter space, fixed points are special, since they are scale invariant. The parameter near a fixed point $\lambda^*$ is called relevant or irrelevant when it flows away or close to $\lambda^*$, respectively. A RG program is to find all fixed points and analyse how the parameters flow near fixed points. One needs to solve the so-called $\beta$ equation
\begin{equation}
\beta_i(\lambda_j) = \frac{\partial \lambda_i}{\partial \mathrm{log}(s)}.
\end{equation}
The zero points of the $\beta$ function are solutions of fixed points of the RG program
\begin{equation}
\beta_i(\lambda_j^*) = 0.
\end{equation}
Near the fixed point, usually one can approimate the $\beta$ function as an linear eigen problem. Eigenvalues of the RG transformation imply how fast $\lambda_i$ flow to or away from $\lambda^*$, which are nothing but the scaling dimensions $\Delta(\tilde{\lambda}_i)$ of the corresponding parameter
\begin{equation}
\frac{\partial \tilde{\lambda}_i}{\partial \mathrm{log}(s)} = \Delta \left(\tilde{\lambda}_i \right) \tilde{\lambda}_i
\end{equation}
where $\tilde{\lambda}_i$ is a linear combination of the orginal paramters (here we shifted the fixed point to be zero and $\tilde{\lambda}_i$ means the distance to the fixed point $\lambda^*_i$). Note that the RG analysis here is also consistent with the renormalizability of a QFT. An irrelevant field ($\Delta\left(\lambda_i\right)<0$) vanishes at IR means it becomes divergent at UV.

There also exist many other RG schemes. For example, one may integrate out all high-energy modes $\vert k \vert > \Lambda$. There will be divergence at low dimensions. A popular way to deal with the divergence is to continue the space dimension $d$ to be a real positive number and make perturbation around the upper or lower critical dimension, which is called as $d \mp \epsilon$ expansion in the literature. Another popular and also elegent RG scheme is to introduce a real space short distance cutoff $a$. The scaling transformation of $a$ is canceled by the change of couplings in the theory. One can use operator product expansion (OPE) to write down the $\beta$ function. In this approach, one only needs to know the OPE coefficients at a known fixed point rather than doing Feynmann diagram calculations.

{\it Example: poor man's scaling of Kondo effect}
{\it Example: perturbative RG analysis of $\phi^4$ theory}
The Ferromagnetic phase transition is usually modeled by a real scalar field theory
\begin{equation}
S = \int d^dx \ \left\{\frac{1}{2}\left(\Delta\phi\right)^2 + \sum_{n=1,2,4}\left(\frac{\lambda_n}{n!}\phi^n\right) \right\}
\end{equation}
where the field $\phi$ can be viewed as flucturations around the mean field solution $\phi_c$ of the action. Following the Wilson-Kadanoff RG scheme, we identify
\begin{equation}
\begin{aligned}
&S_f = \int d^dx \ \left\{\frac{1}{2}\left(\Delta\phi_f\right)^2 + \frac{\lambda_2}{2}\phi_f^2 \right\} \\
&S_s = \int d^dx \ \left\{\frac{1}{2}\left(\Delta\phi_s\right)^2 + \lambda_1 \phi_s + \frac{\lambda_2}{2}\phi_s^2 \right\} \\
&S_c = \int d^dx \ \left\{ \frac{\lambda_4}{4!}\left(\phi_s + \phi_f\right)^4 \right\}.
\end{aligned}
\end{equation}
At one-loop approximation, using cumulant expansion one can find
\begin{equation}
\begin{aligned}
\langle e^{-S_c} \rangle_f = \mathrm{exp}\left\{{-\ExOp{S_c}_f + \frac{1}{2}\left( \ExOp{S_c^2}_f - \ExOp{S_c}_f^2\right)}\right\}
\end{aligned}
\end{equation}%•

• In $\ExOp{S_c}_f$ there is a pure slow mode term $\frac{\lambda_4}{4!}\phi_s^4$ and another one
\begin{equation}
\begin{aligned}
\frac{\lambda_4}{4!} C^2_4 \int d^dx  \phi_s^2 \ExOp{\phi_f(x)\phi_f(x)}_f = \int d^dx \left\{ \frac{\ExOp{\phi_f(x)\phi_f(x)}_f\lambda_4/2}{2}\ \phi_s^2\right\}
\end{aligned}
\end{equation}

• In $\ExOp{S_c^2}_f - \ExOp{S_c}_f^2$ there is one term contributing to the one-loop result
\begin{equation}
\begin{aligned}
\left( \frac{\lambda_4}{4!} C^2_4 \right)^2 \int d^dx \int d^dy \ \left(\phi_s(x)\phi_s(y)\right)^2 \ExOp{\phi_f(x) \phi_f(y)}_f^2
\end{aligned}
\end{equation}

{\it Example: perturbative RG analysis of BKT transition}

\section{Conformal symmetry in $d>2$ dimensions}

\section{Conformal symmetry in two dimension}
In $d=2$ we have infinitely many \emph{local} conformal transformations. The 6 parameter subgroup of conformal transformations that are everywhere well defined is the \emph{global} conformal group $SL(2,\mathbb{C})/\mathbb{Z}_2$. Locally on the algebra level, it becomes the infinite dimensional Witt algebra. In a quantum theory, one can introduce a central extension term and get the famous Virasoro algegra. It is this infinite dimensional symmetry that ensures fields in a cft have nice local properties.

\subsection{Global and local conformal symmetries}

{\it Conformal mappings and Witt algebra}

{\it Conformal Ward identity}

{\it Virasoro algebra: Central extension of Witt algebra}

\subsection{From correlation functions to OPE}
Due to the local nature of field theory, we promote the correlation functions to expansion of non-singular operators, which is termed as operator product expansion (OPE). The idea is basically that far away from the operators inside a bounded region other operators can only feel them as a superposition of single operators (non-singular). The first OPE example follows from the conformal ward identity, which builds up the OPE between the energy momentum tensor and primary fields
\begin{equation}
	T(z) \phi(\omega) \sim \frac{h}{\left(z-\omega\right)^2} \phi(\omega) + \frac{1}{z-\omega} \partial \phi(\omega)
\end{equation}
where on the right hand side, the operators should be understood as to be calculated correlation functions with some other operators located far away from them.  Following the conventions defined in the first chapter, one can easily obtain the OPE of $T(z)$ with $\partial \phi$ for free boson
\begin{equation}
T(z) \partial\phi(\omega) \sim \frac{\partial \phi(\omega)}{\left(z-\omega\right)^2} + \frac{\partial_\omega^2 \phi(\omega)}{z-\omega}
\end{equation}
and $T(z)$ with $\psi$ for free fermion
\begin{equation}
T(z) \psi(\omega) \sim \frac{\frac{1}{2} \psi(\omega)}{\left(z-\omega\right)^2} + \frac{\partial \psi(\omega)}{z-\omega}
\end{equation}

The OPE can be generalized to arbitrary fields
\begin{equation}
A(z) B(\omega) = \sum_{n=-\infty}^{\Delta(A) + \Delta(B)}\  \frac{\{AB\}_n(\omega)}{\left(z-\omega\right)^n}
\end{equation}
where $\{AB\}_n(\omega)$ are non-singular fields. Note that the total scaling dimensions can not be changed in OPE.



\subsection{Energy-momentum tensor and central charge}
The energy-momentum tensor is a quasi-primary field, which does not follow the OPE of $T$ with primaries. There is an aditional term proportional to central charge $c$ in the OPE
\begin{equation}
T(z)T(\omega) \sim \frac{c/2}{\left(z-\omega\right)^4} + \frac{2 T(\omega)}{\left(z-\omega\right)^2} + \frac{\partial T(\omega)}{z-\omega}.
\end{equation}
This term also exists in the conformal transormation of $T$
\begin{equation}
T'(\omega) = \left(\frac{dw}{dz}\right)^{-2} T(z) + \frac{c}{12}\{z;\omega\}
\end{equation}
where $\{z;\omega\}$ denotes the Schwarzian derivative. This is consistent with the fact that this term disappears under global conformal transtions which are true symmetry of CFT.

The central charge $c$ is related to the number of degrees of freedom in the theory. This can be reflected in the calculation of free energy density for a cylinder, which is related to the plane via a conformal transormation
\begin{equation}
\omega = \frac{L}{2\pi} \mathrm{log} (z).
\end{equation}
The energy-momentum tensor becomes
\begin{equation}
T_{cyl}(\omega) = \left(\frac{2\pi}{L}\right)^2 \{T_{pl}(z)\ z^2 - \frac{c}{24}\}.
\end{equation}
The variation of free energy is a response to the change of metric. One can make another coordinate transformation only along with the circumference direction $\omega^0 \rightarrow \omega^0(1+\epsilon)$. Note that this is not a confromal transormation, which will result in the change of the metric tensor. One can find the free energy for a cylinder takes the form of
\begin{equation}
F = f_0 L - \frac{\pi c}{6L},
\end{equation}
which indicates that the conformal anomaly reflects the quantum fluctuation effect to the classifcal conformal symmetry.

\section{Operator formalism}
In this section, we explore the quantization of the cft on a cylinder, which is related to the plane via a conformal transormation.

\subsection{Radial quantization}
On the plane, one has the freedom to choose the direction of space or time for an Euclidean theory. Here we choose the radial direction to be time and the angle direction to be space. A conformal transformation
\begin{equation}
	\xi = \frac{L}{2\pi}\mathrm{log}\left(z\right)
\end{equation}
maps a point $z$ on a complex palne to a point $\xi=t+ix$ on a cylinder with $t$ being the time and $x\in[0,L)$ the space. The Hilbert space defined on the cylinder at a given time $t$ is defined within a circle with a radius $e^{2\pi t/L}$. Naturally the quantum thoery defined on a cylinder can be used to understand the plane.

One can immedialy find many important properties of the radial quantization from the confromal mapping. The time evolution operator, the Hamiltonian, on a cylinder corresponds the dialation operator on the plane and the translation operator, i.e.  the momentum, corresponds to the rotation operator on the plane. Such a quantization scheme for a cft is called radial quantization. The time ordering on a cylinder becomes radial ordering on a plane. As  a conseconce, the commutation of operators for a quantum theory is related to contour integrals through
\begin{equation}
	[A,B] = \oint_0 d\omega \oint_\omega dz \, a(z)b(\omega),
\end{equation}
where $A$ and $B$ are defined as equal time contour integral of local fields. Note that in the coutour integral, we have assumed that there is no other fields existing between the two integral circles, which means the time difference $\epsilon$ here should be infestimal small. In other words, the communicator defined here should be understood as equal-time communicator.

{\it State-field correspondence} Following the quantum theory on a cylinder (an operator in inserted at infinit past time to the vaccum state), we define a state corresponding to the field $\phi(z,\bar{z})$
\begin{equation}
	\vert \phi \rangle = \lim_{z,\bar{z}\to 0}\phi(z,\bar{z})\vert0\rangle.
\end{equation}
Its dual state is defined as
\begin{equation}
	\langle \phi \vert = \lim_{z,\bar{z}\to 0} \bar{z}^{-2h}z^{-2\bar{h}} \langle 0 \vert \phi(1/\bar{z},1/z).
\end{equation}
It is clear that states such definied are properly normalized.

\subsection{Virasoro agebra}
With the radial quantization and state-field correspondence, one can re-express the conformal symmetry, i.e. the Virasoro algera conveniently.  We first introduce quantum operators for local fields and the energy-momentum tensor from equal-time contour integral (or equally mode expansion for local fields)
\begin{equation}
	\phi_n = \frac{1}{2\pi i}\oint dz \, z^{n+h-1} \phi(z),
\end{equation}
in which for the energy-momentum tensor $T(z)$ we denote its mode expansion operator as $L_n$. One then finds operators $L_n$ defind here obey Virasoro algebra using the OPE of $T(z)$ from a straightforward calculation. Again, the communicator between $L_n$ is meaningful as equal-time. One can also obtain the communicator between $L_n$ and $\phi_m$
\begin{equation}
	[L_n, \phi_m] = \left(n(h-1)-m\right) \phi_{n+m}.
\end{equation}
With the Virasoro generators $L_n$, one can also construct states as
\begin{equation}
L_{-k_1}L_{-k_2} \cdots L_{-k_n} \vert \phi \rangle.
\end{equation}

\subsection{The Free Boson}
\subsubsection{Canonical Quantization on the Cylinder}
Let $\phi(x,t)$ be a free Boson field defined on a cylinder of circumference $L$, such that $\phi(x+L,t) = \phi(x,t)$. The Lagrangian of the boson field is
\begin{equation}
  \mathcal{L} = \frac{g}{2}\int dx \left\{(\partial_t\phi)^2-(\partial_x \phi)^2\right\}.
\end{equation}
We can Fourier transform $\phi$ as
\begin{align}
  \phi(x,t) &= \sum_n e^{\frac{2\pi}{L}inx}\phi_n(t), \\
  \phi_n(t) &= \frac{1}{L}\int dx e^{-\frac{2\pi}{L}inx}\phi(x,t).
\end{align}
The Lagrangian can be reexpressed as
\begin{equation}
  \mathcal{L} = \frac{g}{2}\sum_n\left\{\dot{\phi}_n\dot{\phi}_{-n}-\left(\frac{2\pi n}{L}\right)^2\phi_n\phi_{-n}\right\}.
\end{equation}
The momentum conjugate to $\phi_n$ becomes
\begin{equation}
  \pi_n = gL\dot{\phi}_{-n}, \qquad [\phi_n,\pi_m] = i\delta_{nm}.
\end{equation}
The Hamiltonian can be expressed as
\begin{equation}
  H = \frac{1}{2gL}\sum_n\{\pi_n\pi_{-n}+(2\pi ng)^2\phi_n\phi_{-n}\}
\end{equation}
This corresponds to a sum of decoupled harmonic oscillators with frequencies $\omega = \frac{2\pi}{L}\lvert n\rvert$.

We can introduce creation and annihalation operators, which allow the Hamiltonian to be expressed as
\begin{equation}
  H = \frac{1}{2gL}\pi_0^2+\frac{2\pi}{L}\sum_n\left(a_{-n}a_n+\bar{a}_{-n}\bar{a}_n\right).
\end{equation}

The following commutation relation holds
\begin{equation}
  [H,a_{-m}] = \frac{2\pi}{L}ma_{-m}.
\end{equation}
Applying $a_{-m}$ toan eigenstate with energy $E$, creates an eigenstate with energy $E+\frac{2\pi m}{L}$.
The Fourier modes can be expressed as
\begin{equation}
  \phi_n = \frac{i}{n\sqrt{4\pi g}}(a_n-\bar{a}_{-n})
\end{equation}
The fields can be expressed as
\begin{equation}
  \phi(x,t) = \phi_0 + \frac{1}{gL}\pi_0t + \frac{i}{\sqrt{4\pi g}}\sum_{n \neq 0}\frac{1}{n}\left(a_ne^{\frac{2\pi i}{L} n(x-t)}-\bar{a}_{-n}e^{\frac{2\pi i}{L} n(x+t)}\right).
\end{equation}
Transforming to Euclidean space, we can define the following conformal coordinates
\begin{equation}
  z = e^{\frac{2\pi}{L}(\tau-ix)}, \qquad \bar{z} = e^{\frac{2\pi}{L}(\tau+ix)}.
\end{equation}
This results in
\begin{equation}
  \phi(z,\bar{z}) = \phi_0 - \frac{i}{4\pi g}\pi_0\ln(z\bar{z}) + \frac{i}{\sqrt{4\pi g}}\sum_{n \neq 0}\frac{1}{n}\left(a_n z^{-n}+\bar{a}_{n}\bar{z}^{-n}\right).
\end{equation}
The field $\phi$ is not a primary, however the holomorphic field $\partial \phi$ is.
\begin{equation}
  i \partial \phi(z)  = \frac{\pi_0}{4\pi g z}+\frac{i}{\sqrt{4\pi g}}\sum_{n \neq 0} a_n z^{-n-1}
\end{equation}
\subsubsection{Vertex Operators}
There exists an infinite variety of local fields related to $\phi$ without introducing a scale. These are called the vertex operators $\mathcal{V}_\alpha$.
\begin{equation}
  \mathcal{V}_\alpha = :e^{i\alpha\phi(z,\bar{z})}:
\end{equation}
The vertex operators have conformal dimensions $h(\alpha)=\bar{h}(\alpha) = \frac{\alpha^2}{8\pi g}$.
\subsection{The Fock Space}
The eigenstates of $H$ can be labeled by the eigenvalues of $\pi_0$. This means that the Fock space is built upon a one-parameter family of vacua $\lvert{\alpha}\rangle$.

We know that $T(z)$ is given by
\begin{align}
  T(z) &= -2\pi g :\partial \phi(z) \partial\phi(z) :\\
  &= \frac{1}{2} \sum_{n,m}z^{-n-m-2}:a_na_m: .
\end{align}
From this we can derive the expression for the Virasoro operators
\begin{align}
  L_n &= \frac{1}{2}\sum_{m\in\mathbb{Z}}a_{n-m}a_m \quad(n\neq 0),\\
  L_0 &= \sum_{n>0}a_{-n}a_n + \frac{1}{2}a_0^2.
\end{align}
This allows for the Hamiltonian to be expressed as
\begin{equation}
  H = \frac{2\pi}{L}(L_0+\bar{L}_0)
\end{equation}
Furthermore the elements of the Fock space $a_{-1}^{n_1}a_{-2}^{n_2}...\bar{a}_{-1}^{m_1}\bar{a}_{-2}^{m_2}...\lvert\alpha\rangle$ are eigenstates of $L_0$ with conformal dimensions $h=\frac{1}{2}\alpha^2+\sum_jjn_j$ and $\bar{h}=\frac{1}{2}\alpha^2+\sum_jjm_j$.
The different vacua $\lvert \alpha\rangle$ are related to the absolute vacuum $\lvert 0\rangle$ by the vertex operators $\mathcal{V}_\alpha$.
\subsubsection{Twisted Boundary Conditions}
We can also assume anti-periodic boundary condition. This is compatible with the Lagrangian because it is quadratic in the fields. Changing to anti-periodic boundaries makes the summation index half-integer valued and removes the zero mode. There are now two vacua $\lvert 0_+ \rangle$ and $\lvert 0_- \rangle$.

We have that
\begin{equation}
    \langle\phi\partial\phi\rangle = \frac{1}{w}\sum_{n>0}\left(\frac{w}{z}\right)^n.
\end{equation}
In the periodic case the sum is over integer values and becomes
\begin{equation}
  \langle\phi\partial\phi\rangle = \frac{1}{z-w}.
\end{equation}
In the anti-periodic case the sum is over half integer values and becomes
\begin{equation}
  \langle\phi\partial\phi\rangle = \sqrt{\frac{z}{w}}\frac{1}{z-w}.
\end{equation}

For the vacuum expactation value of the energy-momentum tensor, $\langle T(z)\rangle$, we have in the periodic case $\langle T(z)\rangle = 0$, but in the anti-periodic case $\langle T(z)\rangle=\frac{1}{16z^2}$.

\subsubsection{Compactified Boson}
We can identify $\phi$ with $\phi+2\pi R$ to get the compact boson. In general we can consider the boundary condition
\begin{equation}
  \phi(x+L,t) = \phi(x,t) + 2\pi m R,
\end{equation}
where $m$ represents the winding number of the field. This modifies the mode expansion as
\begin{equation}
  \phi(x,t) = \phi_0 + \frac{n}{gRL}t + \frac{2\pi m R}{L}x + \frac{i}{\sqrt{4\pi g}}\sum_{k\neq 0}\frac{1}{k}\left(a_ke^{\frac{2\pi i k}{L}(x-t)}-\bar{a}_{-k}e^{\frac{2\pi i k}{L}(x+t)}\right).
\end{equation}
After reexpressing in complex coordinates and taking the derivative we get
\begin{equation}
  i\partial\phi(z) = (\frac{n}{4\pi g R}+\frac{1}{2}mR)\frac{1}{z}+\frac{1}{\sqrt{4\pi g}\sum_{k\neq 0}a_k z^{-k-1}}.
\end{equation}
The virasoro operators $L_0$ and $\bar{L}_0$ can be expressed as
\begin{align}
  L_0 &= \sum_{n>0}a_{-n}a_n + 2\pi g \left(\frac{n}{4\pi g R}+\frac{1}{2}mR\right)^2\\
  \bar{L}_0 &= \sum_{n>0}\bar{a}_{-n}\bar{a}_n + 2\pi g \left(\frac{n}{4\pi g R}-\frac{1}{2}mR\right)^2\\
\end{align}
\subsection{The Free Fermion}
The free fermion action is given by
\begin{equation}
  S = \frac{1}{2}g\int d^2 x \Psi^\dagger\gamma^0\gamma^\mu\partial_\mu\Psi
\end{equation}
The central charge of this theory is $c=1/2$ and $\psi$ has as conformal dimension $h=1/2$.
\subsubsection{Canonical Quantization on a Cylinder}
We can take the mode expansion of $\psi$ at $t=0$ on a cylinder with circumference $L$. This gives
\begin{equation}
  \psi(x) = \sqrt{\frac{2\pi}{L}}\sum_kb_ke^{\frac{2\pi i}{L}k x}.
\end{equation}
There are two possible types of boundary conditions. With the periodic (Ramond) boundary conditions the index k takes on integer values. With anti-periodic (Neveu-Schwarz) boundary conditions the index k must take half-integer values.

The Hamiltonian can be written as
\begin{equation}
  H = \sum_k>0\omega_k b_{-k}b_k +E_0, \qquad \omega_k = \frac{2\pi\lvert k\rvert}{L}.
\end{equation}
\subsubsection{Mapping onto the Plane}
Mapping $\psi$ to the plane gives
\begin{equation}
  \psi_{cyl}(z) = \sqrt{\frac{2\pi z}{L}}\psi_{pl}(z)
\end{equation}
and thus
\begin{equation}
  \psi(z) = \sum_k b_k z^{-k-\frac{1}{2}}
\end{equation}
This transformation swaps the boundary Conditions
\begin{align}
  \psi(e^{2\pi i}z) &=\psi(z) \qquad(Ramond)\\
  \psi(e^{2\pi i}z) &\psi(z) \qquad(Neveu-Schwarz)\\
\end{align}

The different sectors will have a different two-point correlation function. For the NS sector we have
\begin{equation}
  \langle\psi(z)\psi(w)\rangle = \frac{1}{z-w}.
\end{equation}
In the R sector we have
\begin{equation}
  \langle\psi(z)\psi(w)\rangle = \frac{1}{2}\frac{\sqrt{z/w}+\sqrt{w/z}}{z-w}.
\end{equation}
Furthermore depending on the boundary conditions the energy-momentum tensor will gain a non-zero expectation value.
\begin{align}
  \langle T(z)\rangle = 0 \qquad(Neveu-Schwarz)\\
  \langle T(z)\rangle = \frac{1}{16z^2} \qquad(Ramond)\\
\end{align}
\subsubsection{Vacuum Energies}
The energy momentum tensor on the plane can be written as
\begin{equation}
  T(z) = \frac{1}{2}\sum_{n,k}(k+\frac{1}{2})z^{-n-2}:b_{n-k}b_k:,
\end{equation}
which naturally leads to
\begin{equation}
  L_n = \frac{1}{2}\sum_k(k+\frac{1}{2}):b_{n-k}b_{k}:
\end{equation}
$L_0$ is given by different expressions depending on the boundary conditions.
\begin{align}
  L_0 &= \sum_{k>0}kb_{-k}b_k \qquad (NS)\\
  L_0 &= \sum_{k>0}kb_{-k}b_k + \frac{1}{16} \qquad (R)\\
\end{align}
From this we can express the Hamiltonian as
\begin{equation}
  H = \frac{2\pi}{L}(L_0 + \bar{L}_0-\frac{c}{12}).
\end{equation}
\section{Boundary cft}
CFT can also be defined in a manifold with boundaries, in which the nice local properties are still applied from the CFT defined on a plane. A boundary cft can be expressed as a cft defined on the upper half plane with a real axis at its bottom, at which fields obey conformal boundaries. A boundary changing operator is inserted at the origin point if the left and right half plane have different boundaries. This whole subject is related to a name, Cardy. He adopted the method of mirror image to map the anti-holomorphic part of the theory to the lower half plane, which resultes in a chiral cft defined on the whole plane.  Now the conformal symmetry realized on the plane can be used to study boundary cfts. For example, the two-point correlation functions becomes four-point chiral correlation functions in a boundary cft. The classification of boundary cfts is also closely realted to surface critical behaviors. Different surface critical behaviors are naturally understood as different conformal boundaries.


\appendix

\section{Central extensions of Lie algebras}
In this section $\mathfrak{g},\mathfrak{h},...$ denote (possibly infinite) Lie algebras over some field $\mathbb{K} = \mathbb{R}, \mathbb{C}$. This section is mainly based on Wikipedia and~\cite{Schottenloher}.
\subsection{Extensions}
{\bf Definition:} A \emph{Lie algebra extension} is a short exact sequence of Lie algebras:
\begin{equation}
	\mathfrak{h}\overset{\iota}{\rightarrow}\mathfrak{e}\overset{\pi}{\rightarrow}\mathfrak{g}.
\end{equation}
One calls $\mathfrak{e}$ an extension of $\mathfrak{g}$ by $\mathfrak{h}$. By exactness of the sequence one has $\mathfrak{g}\cong\mathfrak{e}\slash\Ima\iota$.\\

{\bf Definition:} A \emph{central extension} is an extension $\mathfrak{e}$ of $\mathfrak{g}$ by $\mathfrak{h}$, such that $\Ima \iota$ is contained in the center of $\mathfrak{e}$, $\iota(\mathfrak{h})\subseteq Z(\mathfrak{e})$.\\

Notice that for a central extension $\mathfrak{h}$ is necessarily abelian. We now introduce a notion of trivial central extensions as follows:

{\bf Definition:} A Lie algebra extension
\begin{equation}
	\mathfrak{h}\overset{\iota}{\rightarrow}\mathfrak{e}\overset{\pi}{\rightarrow}\mathfrak{g}
\end{equation}
\emph{splits} if there exists a Lie algebra morphism $\beta: \mathfrak{g}\mapsto\mathfrak{e}$ such that $\pi\circ\beta = \text{id}_{\mathfrak{e}}$. $\beta$ is called a splitting map.\\

A central extension
\begin{equation}
	\mathfrak{h}\overset{\iota}{\rightarrow}\mathfrak{e}\overset{\pi}{\rightarrow}\mathfrak{g}.
\end{equation}
that splits is trivial in the sense that it is equivalent\footnote{To do: introduce the notion of equivalent extensions.} to one where $\mathfrak{e}\cong\mathfrak{g}\oplus\mathfrak{h}$.\\

Let us now consider a central extension and a map (not necesserily a Lie algebra homomorphism) $\beta:\mathfrak{g}\rightarrow\mathfrak{e}$ such that $\pi\circ\beta = \text{id}_{\mathfrak{e}}$. From this map construct
$\Theta: \mathfrak{g}\times\mathfrak{g}\rightarrow\mathfrak{h}$ as follows:
\begin{equation}
	\Theta(x,y) := \left[\Theta(x),\Theta(y)\right] - \Theta\left([x,y]\right).
\end{equation}
This map is:
\begin{enumerate}
	\item \label{prop:antisym} Antisymmetric.
	\item \label{prop:bilinear} Bilinear.
	\item \label{prop:Jacobi} Satisfies $\Theta(x,[y,z]) + \Theta(y,[z,x]) + \Theta(z,[x,y]) = 0$.
\end{enumerate}
Given $\Theta$ one can now show that there is an isomorphism between the vector spaces $\mathfrak{e}\cong\mathfrak{g}\oplus\mathfrak{h}$ that is given by:
\begin{equation}
	\Psi:\mathfrak{g}\oplus\mathfrak{h}\mapsto\mathfrak{e}:(x,y)\mapsto\beta(x) + y.
\end{equation}
A Lie bracket on $\mathfrak{g}\oplus\mathfrak{h}$ is given by:
\begin{equation}
	[x\oplus z, y\oplus z']_{\mathfrak{e}} := [x,y]_\mathfrak{g} + \Theta(x,y).
\end{equation}

{\bf Lemma:} In the above construction $\beta$ is a splitting map if and only if
\begin{equation}
	\Theta(x,y) = \mu([x,y]),
\end{equation}
for some $\mu\in\text{Hom}(\mathfrak{g},\mathfrak{h})$.\\

Now comes the classification of central extensions of Lie algebras:\\

{\bf Theorem:} Every central extension comes from a map $\Theta$ that satisfies the above properties (\ref{prop:antisym}-\ref{prop:Jacobi}). Conversely, every central extension gives rise to a map $\Theta$ that satisfies the above properties (\ref{prop:antisym}-\ref{prop:Jacobi}).

\subsection{Lie algebra cohomology}
The classification of Lie algebra extensions is very satisfying. It smells a lot like a cohomological classification. Indeed, the extensions are classified by functions depending on two variables satisfying the condition (\ref{prop:Jacobi}) that is exactly the one needed to fulfill the Jacobi identity of the central extension. Moreover, the central extension is trivial if the \emph{2-cocycle} $\Theta$ is trivial in the following sense: $\Theta(x,y) = \mu([x,y])$. This is reminiscent of considering 2-cocycles to be trivial if they are equal to a coboundary. Let us put this on a bit more rigorous footing.\\

{\bf Definitions:}
\begin{enumerate}
	\item $Z^2(\mathfrak{g},\mathfrak{h}) = \left\{\Theta\in\Lambda^2(\mathfrak{g},\mathfrak{h})|\Theta:(\ref{prop:Jacobi})\right\}$.
	\item $B^2(\mathfrak{g},\mathfrak{h})=\left\{\Theta:\mathfrak{g}\times\mathfrak{g}\mapsto\mathfrak{h}|\exists\mu\in\text{Hom}(\mathfrak{g},\mathfrak{h}):\Theta(-,-)=\mu([-,-])\right\}$.
	\item $H^2(\mathfrak{g},\mathfrak{h}):=Z^2(\mathfrak{g},\mathfrak{h})/B^2(\mathfrak{g},\mathfrak{h})$.
\end{enumerate}
$H^2$ is of course called the second cohomology group. We thus obtain the following reformulation of the classification of central extensions:\\

{\bf Theorem:} The equivalence classes of central extensions
\begin{equation}
	\mathfrak{h}\overset{\iota}{\rightarrow}\mathfrak{e}\overset{\pi}{\rightarrow}\mathfrak{g}
\end{equation}
are in one-to-one correspondence with the elements of $H^2(\mathfrak{g},\mathfrak{h})$.\\

For completeness, let us introduce a notion of cochain complexes for Lie algebras. A cochain $f$ is a alternating multilinear map $f$:
\begin{equation}
	f:\Lambda^n \mathfrak{g} \mapsto \mathfrak{h}.
\end{equation}
Here, $\mathfrak{h}$ is considered a $\mathfrak{g}$-module or -representation.

The differential of an $n$-cochain is given by
\begin{equation}
\begin{aligned}
	(d f)\left(x_1, \ldots, x_{n+1}\right) =
	&\sum_i     (-1)^{i+1}x_i\, f\left(x_1, \ldots, \hat x_i, \ldots, x_{n+1}\right) + \\
	&\sum_{i<j} (-1)^{i+j}      f\left(\left[x_i, x_j\right], x_1, \ldots, \hat x_i, \ldots, \hat x_j, \ldots, x_{n+1}\right)\, ,
\end{aligned}
\end{equation}

so for example, with trivial action we obtain
\begin{equation}
	(df)(x_1,x_2) = f([x_1,x_2]),
\end{equation}
and
\begin{equation}
\begin{aligned}
	(df)(x_1,x_2,x_3) &= -f([x_1,x_2],x_3) + f([x_1,x_3],x_2) - f([x_2,x_3],x_1)\\
	&= -f([x_1,x_2],x_3) - f([x_3,x_1],x_2) - f([x_2,x_3],x_1)\\
	&= f(x_3,[x_1,x_2]) + f(x_2,[x_3,x_1]) + f(x_1,[x_2,x_3]).
\end{aligned}
\end{equation}
So clearly, $Z^2(\mathfrak{g},\mathfrak{h})$ defined above is the group of 2-cocycles satisfying $d\Theta=0$ and $B^2(\mathfrak{g},\mathfrak{h})$ the set of coboundaries: $\Theta = d\mu$.
\bibliography{refs.bib}

\nolinenumbers

\end{document}
